\documentclass[12pt]{article}
\usepackage{titlesec, amsmath, amsthm, amssymb, verbatim, color, graphics, geometry, mdframed, fancyhdr, courier, graphicx, multicol, enumitem, makecell}

\usepackage[french]{babel}
\usepackage[T1]{fontenc}
\usepackage[dvipsnames]{xcolor}
\graphicspath{{/home/ash/moci/uml/Images}}
\geometry{tmargin=.75in, bmargin=.75in, lmargin=.75in, rmargin = .75in}
\setlength{\headheight}{15pt}

\newtheorem{thm}{Théorème}[section]
\theoremstyle{definition}
\newtheorem{ex}{Exercice}
\theoremstyle{definition}
\newtheorem{defn}{Définition}
\theoremstyle{remark}
\newtheorem{rem}{Remarque}
\theoremstyle{remark}
\newtheorem{example}{Example}
\theoremstyle{definition}
\newtheorem{prop}{Proposition}
\theoremstyle{remark}
\newtheorem{cexample}{Contre-example}

% \graphicspath{ {/home/ash/Images/cours} }

\title{Rapport Coding Week}
\author{Lowell, Raphael, Ana, Ingrid}
\date{\today}

\pagestyle{fancy}
\begin{document}


\maketitle
\tableofcontents
\newpage
\fancyhead[L]{Roadmap}
\fancyhead[C]{\textbf{CodeNames}}
\fancyhead[R]{TN A2}

\section{Roadmap} 

\subsection{\today}

\begin{enumerate}
    \item Créer un projet gradle pour avoir une base commune
    \item Configurer le répo git
    \item Assigner les rôles
    \item Créer le layout de l'application
    \item Créer une première grille du jeu, avec une taille par defaut
    \item Etablir les principales classes, l'architecture du jeu
\end{enumerate}





\end{document}


